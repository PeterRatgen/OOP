\documentclass{article}

\usepackage[utf8]{inputenc}
\usepackage[danish]{babel}
\usepackage{float}
\usepackage{fancyhdr}
\usepackage{amsmath}
\usepackage{color}
\usepackage{listings}
\usepackage{graphicx}
\usepackage{lastpage}
\usepackage{enumitem}
\usepackage[a4paper, top = 1in, bottom = 1in, left=1in,right=1in]{geometry}
\usepackage{tikz}
\usepackage{tikz-qtree}
\usepackage{listingsutf8}

\title{Noter til Objektorienteret programmering}
\author{Peter Heilbo Ratgen}
\date{\today, 1. semester}


\begin{document}
  \maketitle
  \newpage
  \tableofcontents
  \newpage
  \section{Intro - 3. september}
  Et modul er en lektion og en øvelsestime. Der er to moduler på en uge, men det
  sidste moduls øvelsestime ligger efter en weekend. Den første øvelsestime
  ligger næste dag efter forlæsingen.
  Han anbefaler efter en forlæsning at man læser materialet, ser
  screencastvideoerne. Vi skal ikke have lavet øvelserne til selve øvelsestimen.
  Øvelsestimerne er selvfølgelig ledet af en instruktor. Opgaverne i
  øvelsestimerne er peerbaseret, hvor man har en rolle A og en rolle B. Hvor den
  ene dikterer hvad den anden skal skrive. Hvis man ikke kan det må man snakke
  sammen om det. Det vigtigste er at få noget kode i fingrene. Det er let at
  falde igennem på det.

  \paragraph{Evaluering}
  Vi har tællende aktiviter, som er 3 stk som tæller 30 procent. Det er multiple
  choice eller multiple answer. Der kan også være essay-style spørgsmål. Hvis
  man misser alle 3 kan det være svært at få en god karakter. Man er lidt nødt
  til at læse op til hver af de her tællende aktiviter.
  Der er en mundtlig fagprøve. Vi kan bruge noter, udleveret litteratur, slides,
  noter. Den vægter med 70\%. Man trækker en ud af 4 cases, de dækker over nogle
  meget brede emner. Der selvfølgelig spørgsmål til pensum af lærer og censor.
  Det handler om at blive gode til håndværket og blive god til stoffet, end at
  det handler om at få en god karakter.
  \begin{enumerate}
    \item Krav
    \item Analyse og design
    \item Implementering
    \item Test
    \item Deployment
    \item Vedligehold
  \end{enumerate}
  Plus andre værktøjer. Her fokuserer vi på implementering, og berører Analyse
  og design og test. Der kommer også en smule med kravindfangning.

  Vi bliver bedømt på det der står som faglige mål til fagbeskrivelsen.
  
  \paragraph{Objektorienteret programmering}
  er et paradigme inden for programmering. Det er en måde at modellere verden
  på. Kurset bruger Java, vi kunne også bruge C++, C\# og Smalltalk. Vi bruger
  det fordi det er platformsuafhængigt. Der er godt miljøer  til programmering
  vi bruger IntelliJ. Bruges til forskning på universitet. Et af de mest
  anvendte programmeringssprog. 
  

\end{document}
